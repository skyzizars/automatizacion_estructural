\documentclass[12pt]{article}
\usepackage[spanish,es-tabla]{babel}
\usepackage{graphicx}
\usepackage{wrapfig}
\usepackage[a4paper]{geometry}
\usepackage{amsmath}
\usepackage{mathtools}
\usepackage{multicol}


\geometry{top=1.5cm, bottom=1.5cm, left=1.25cm, right=1.25cm}
\newcommand*{\grabto}[2]{\IfFileExists{#2}{}{\immediate\write18{curl \detokenize{#1 -o #2}}}}

\begin{document}
    \subsection{Diseño de columnas}
        Diseñaremos la columna en el eje A con interseción del eje 5:
        \textbf{Geometría de la columna:}
        \begin{itemize}
            \item Ancho: b = 30 cm
            \item Peralte: h = 65 cm            
        \end{itemize}
        \textbf{Datos del refuerzo:}
        \begin{itemize}
            \item Recubrimiento: $r_e$ = 4 cm
            \item Diámetro del acero longitudinal: $d_b$ = 5/8"
            \item Diámetro de los estribos: $d_e$ = 3/8"
        \end{itemize}
        \textbf{Factores de minoración:}
        \begin{itemize}
            \item Factor de minoración a compresión: $\phi_{c}$ = 0.70 (Art 9.3.2.2)
            \item Factor de minoración a flexocompresión: $\phi_{fc}$ = 0.70 - 0.90 ($\phi$ incrementa linealmente de 0,7 a 0,9 a medida que $\phi Pn$ disminuye desde 0,1 $f'_cAg$ o $\phi Pb$, el que sea menor, hasta cero - Art 9.3.2.2)
        \end{itemize}
        
        \subsubsection{Diseño por flexión y carga axial:}
        Un elemento es considerado como columna cuando la carga axial amplificada en compresión $P_u$ excede $0.1 \cdot f'_c Ag$. (Art 21.5.1.1 y 21.6.11)\\
        La cuantía mínima para elementos en compresión no debe ser menor que 1\% (Art. 10.9.1)\\
        La resistencia máxima de diseño a compresión pura será: (Ecu. 10.2)
        \[ \phi P_n = 0.80 \cdot \phi_c \cdot [0.85 \cdot f'_c \cdot (Ag - A_{st}) + f_y \cdot A_{st} ] \]

        \subsubsection{Resistencia a corte}
        La resistencia a cortante del concreto a compresión y tracción está dado respectivamente por:
        \begin{align*}
            &\phi V_c = \phi_c \cdot 0.53 \cdot \sqrt{f'_c} \cdot b \cdot d \cdot \left( 1 + \frac{N_u}{140 \cdot A_g} \right)   \hspace{2cm} & \text{(Ecu 11.4 E-060)}\\ 
            &\phi V_c = \phi_c \cdot 0.53 \cdot \sqrt{f'_c} \cdot b \cdot d \cdot \left( 1 - \frac{N_u}{35 \cdot A_g} \right) & \text{(Ecu 11.8 E-060)}
        \end{align*}

        Donde:
        $N_u$: Carga Axial \\
        $A_g$: Área bruta de la sección \\
        d: Peralte efectivo en la dirección de análisis \\
        b: Ancho de la columna en la dirección de análisis \\
        Los peraltes efectivos en las direcciones fuerte y débil de la columna serán: 
        \begin{align*}
            d_x = h_c - r_e - d_e - d_b/2\\
            d_y = c_c - r_e - d_e - d_b/2
        \end{align*}
        Las capacidades máximas de los estribos a corte en las direcciones fuerte y débil de la columna serán:
        \begin{align*}
            V_{ex,max} = 2.1 \cdot \sqrt{210} \cdot 65 \cdot 59.26 = 117.22ton \\
            V_{ey,max} = 2.1 \cdot \sqrt{210} \cdot 30 \cdot 24.26 = 117.22ton
        \end{align*}

        \begin{tabular}{cccccccccc}
                \multicolumn{10}{l} {Combinación: 1.25 (CM+CV) +SY:}\\
                \hline
                Piso & h (m) & Pu (ton) & a (cm) & Mn (ton.m) & Vu (ton) & $\phi$Vc (ton) & Ve & ramas & s (cm) \\
                \hline
                1 & 2 & 3 & 4 & 5 & 6 & 7 & 8 & 9 & 10 \\
                1 & 2 & 3 & 4 & 5 & 6 & 7 & 8 & 9 & 10 \\
                1 & 2 & 3 & 4 & 5 & 6 & 7 & 8 & 9 & 10 \\
                1 & 2 & 3 & 4 & 5 & 6 & 7 & 8 & 9 & 10 \\
                1 & 2 & 3 & 4 & 5 & 6 & 7 & 8 & 9 & 10 \\
                1 & 2 & 3 & 4 & 5 & 6 & 7 & 8 & 9 & 10 \\
                1 & 2 & 3 & 4 & 5 & 6 & 7 & 8 & 9 & 10 \\
                1 & 2 & 3 & 4 & 5 & 6 & 7 & 8 & 9 & 10 \\
                1 & 2 & 3 & 4 & 5 & 6 & 7 & 8 & 9 & 10 \\
        \end{tabular}
        
        \subsubsection{Refuerzo transversal en zonas de confinamiento}
        Longitud de confinamiento será la menor de: (Art. 21.4.5.3 y 21.6.4.4 E-060)
        \begin{enumerate}
            \item La sexta parte de la luz libre de la columna: $h_n/6$
            \item La mayor dimensión del elemento: max($b_c,h_c$)
            \item 50 cm
        \end{enumerate}
        El espaciamiento de refuerzo transversal dentro de la zona de confinamiento será el menor de: (Art. 21.4.5.3 y 21.6.4.2 E-060)

        \begin{center}
            \begin{tabular}{ll}
                \hline
                Sistemas de muros & Sistemas de pórticos duales \\
                \hline
                $\bullet 8d_b$ & $\bullet 6d_b$ \\
                $\bullet min(b,h)/2$ & $\bullet min(b/h)/3$ \\
                $\bullet 10 cm$ & $\bullet 10 cm$ \\
                \hline
            \end{tabular}
        \end{center}

        El espaciamiento del refuerzo transversal fuera de la zona de confinamiento será el menor de: (Art. 21.4.5.4 y 21.6.4.5 E-060)

       \begin{center}
            \begin{tabular}{ll}
                \hline
                Sistemas de muros & Sistemas de pórticos duales \\
                \hline
                $\bullet 16d_b$ &  \\
                $\bullet 48d_e$ & $\bullet 10d_b$ \\
                $\bullet 30 cm$ & $\bullet 25 cm$ \\
                $\bullet min(b,h)$  \\
                \hline
            \end{tabular}
        \end{center}

        Refuerzo transversal mínimo en columnas:
        \begin{align*}
            &A_{sh} = 0.3\frac{s \cdot b_c \cdot f'_c}{f_{yh}} \left[ \frac{A_g}{A_ch} -1  \right] &(a)\\
            &A_{sh} = 0.09 \frac{s \cdot b_c \cdot f'_c}{f_{yh}} &(b)
        \end{align*}

        Donde: \\
        $A_{sh}$: Área de refuerzo transversal\\
        $b_c$: dimensión del núcleo confinado del elemento normal al refuerzo medida centro a centro del refuerzo de confinamiento\\
        $A_{ch}$: área del núcleo confinado medida al exterior del refuerzo de confinamiento\\
        $A_g$: área de la sección bruta de la columna\\
        $s$: separación del refuerzo transversal\\
        $f_{yh}$: esfuerzo de fluencia del refuerzo transversal\\

        Nota:\\
        Cuando la resistencia de diseñno del núcleo de la sección transversal del elemento satisface
        los requisitos de las combinaciones de carga de diseño, incluyendo el efecto sísmico, no
        es necesario satisfacer la ecuación (a)\\

        \subsubsection{Requisitos adicionales para columnas de sistamas de pórticos o duales}
        Los elementos estructurales que cumplan con las siguientes condiciones 
        \begin{itemize}
            \item La carga axial amplificada excede $0.1 \cdot f'_c \cdot A_g = 40.94$ ton
            \item La menor dimensión debe ser mínimamente B=25cm
            \item La relación entre la menor dimensión y la mayor será menor que B/L $\geq$ 0.25      
        \end{itemize}

        \textbf{Resistencia mínima a flexión en las columnas:}\\
        Las resistencias a flexión en las columnas en las caras de los nudos deben satisfacer la siguiente ecuación: (Art. 21.6.6.2 E-060)\\
        \[\Sigma M_{nc} \geq 1.2 \Sigma M_{nv}\]
        Donde:
        $M_{nc}$:suma de los momentos nominales de flexión de las columnas que llegan al nudo, evaluados en las caras del nudo. La resistencia a la flexión de la columna debe calcularse para la fuerza axial amplificad, consistente con la direccion de las fuerzas laterales consideradas, que conduzca a la resistencia a la flexión más baja\\
        $M_{nv}$: suma de los momentos resistentes nominales a flexión de las vigas que llegan al nudo evaluados en las caras del nudo.

         



        
\end{document}