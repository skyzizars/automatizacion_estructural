\documentclass[12pt]{article}
\usepackage[spanish,es-tabla]{babel}
\usepackage{graphicx}
\usepackage{wrapfig}
\usepackage[a4paper]{geometry}
\usepackage{amsmath}
\usepackage{mathtools}
\usepackage{multicol}


\geometry{top=1.5cm, bottom=1.5cm, left=1.25cm, right=1.25cm}
\newcommand*{\grabto}[2]{\IfFileExists{#2}{}{\immediate\write18{curl \detokenize{#1 -o #2}}}}

\begin{document}
    \subsection{Diseño de columnas}
        Diseñaremos la columna en el eje A con interseción del eje 5:
        \textbf{Geometría de la columna:}
        \begin{itemize}
            \item Ancho: b = 30 cm
            \item Peralte: h = 65 cm            
        \end{itemize}
        \textbf{Datos del refuerzo:}
        \begin{itemize}
            \item Recubrimiento: $r_e$ = 4 cm
            \item Diámetro del acero longitudinal: $d_b$ = 5/8"
            \item Diámetro de los estribos: $d_e$ = 3/8"
        \end{itemize}
        \textbf{Factores de minoración:}
        \begin{itemize}
            \item Factor de minoración a compresión: $\phi_{c}$ = 0.70 (Art 9.3.2.2)
            \item Factor de minoración a flexocompresión: $\phi_{fc}$ = 0.70 - 0.90 ($\phi$ incrementa linealmente de 0,7 a 0,9 a medida que $\phi Pn$ disminuye desde 0,1 $f'_cAg$ o $\phi Pb$, el que sea menor, hasta cero - Art 9.3.2.2)
        \end{itemize}
        
        \subsubsection{Diseño por flexión y carga axial:}
        Un elemento es considerado como columna cuando la carga axial amplificada en compresión $P_u$ excede $0.1 \cdot f'_c Ag$. (Art 21.5.1.1 y 21.6.11)\\
        La cuantía mínima para elementos en compresión no debe ser menor que 1\% (Art. 10.9.1)\\
        La resistencia máxima de diseño a compresión pura será: (Ecu. 10.2)
        \[ \phi P_n = 0.80 \cdot \phi_c \cdot [0.85 \cdot f'_c \cdot (Ag - A_{st}) + f_y \cdot A_{st} ] \]

        \subsubsection{Resistencia a corte}
        

        



\end{document}